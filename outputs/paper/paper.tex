% Options for packages loaded elsewhere
\PassOptionsToPackage{unicode}{hyperref}
\PassOptionsToPackage{hyphens}{url}
%
\documentclass[
]{article}
\title{My title\thanks{Code and data are available at: LINK.}}
\usepackage{etoolbox}
\makeatletter
\providecommand{\subtitle}[1]{% add subtitle to \maketitle
  \apptocmd{\@title}{\par {\large #1 \par}}{}{}
}
\makeatother
\subtitle{My subtitle if needed}
\author{Luxuan Zhu}
\date{29 January 2022}

\usepackage{amsmath,amssymb}
\usepackage{lmodern}
\usepackage{iftex}
\ifPDFTeX
  \usepackage[T1]{fontenc}
  \usepackage[utf8]{inputenc}
  \usepackage{textcomp} % provide euro and other symbols
\else % if luatex or xetex
  \usepackage{unicode-math}
  \defaultfontfeatures{Scale=MatchLowercase}
  \defaultfontfeatures[\rmfamily]{Ligatures=TeX,Scale=1}
\fi
% Use upquote if available, for straight quotes in verbatim environments
\IfFileExists{upquote.sty}{\usepackage{upquote}}{}
\IfFileExists{microtype.sty}{% use microtype if available
  \usepackage[]{microtype}
  \UseMicrotypeSet[protrusion]{basicmath} % disable protrusion for tt fonts
}{}
\makeatletter
\@ifundefined{KOMAClassName}{% if non-KOMA class
  \IfFileExists{parskip.sty}{%
    \usepackage{parskip}
  }{% else
    \setlength{\parindent}{0pt}
    \setlength{\parskip}{6pt plus 2pt minus 1pt}}
}{% if KOMA class
  \KOMAoptions{parskip=half}}
\makeatother
\usepackage{xcolor}
\IfFileExists{xurl.sty}{\usepackage{xurl}}{} % add URL line breaks if available
\IfFileExists{bookmark.sty}{\usepackage{bookmark}}{\usepackage{hyperref}}
\hypersetup{
  pdftitle={My title},
  pdfauthor={Luxuan Zhu},
  hidelinks,
  pdfcreator={LaTeX via pandoc}}
\urlstyle{same} % disable monospaced font for URLs
\usepackage[margin=1in]{geometry}
\usepackage{color}
\usepackage{fancyvrb}
\newcommand{\VerbBar}{|}
\newcommand{\VERB}{\Verb[commandchars=\\\{\}]}
\DefineVerbatimEnvironment{Highlighting}{Verbatim}{commandchars=\\\{\}}
% Add ',fontsize=\small' for more characters per line
\usepackage{framed}
\definecolor{shadecolor}{RGB}{248,248,248}
\newenvironment{Shaded}{\begin{snugshade}}{\end{snugshade}}
\newcommand{\AlertTok}[1]{\textcolor[rgb]{0.94,0.16,0.16}{#1}}
\newcommand{\AnnotationTok}[1]{\textcolor[rgb]{0.56,0.35,0.01}{\textbf{\textit{#1}}}}
\newcommand{\AttributeTok}[1]{\textcolor[rgb]{0.77,0.63,0.00}{#1}}
\newcommand{\BaseNTok}[1]{\textcolor[rgb]{0.00,0.00,0.81}{#1}}
\newcommand{\BuiltInTok}[1]{#1}
\newcommand{\CharTok}[1]{\textcolor[rgb]{0.31,0.60,0.02}{#1}}
\newcommand{\CommentTok}[1]{\textcolor[rgb]{0.56,0.35,0.01}{\textit{#1}}}
\newcommand{\CommentVarTok}[1]{\textcolor[rgb]{0.56,0.35,0.01}{\textbf{\textit{#1}}}}
\newcommand{\ConstantTok}[1]{\textcolor[rgb]{0.00,0.00,0.00}{#1}}
\newcommand{\ControlFlowTok}[1]{\textcolor[rgb]{0.13,0.29,0.53}{\textbf{#1}}}
\newcommand{\DataTypeTok}[1]{\textcolor[rgb]{0.13,0.29,0.53}{#1}}
\newcommand{\DecValTok}[1]{\textcolor[rgb]{0.00,0.00,0.81}{#1}}
\newcommand{\DocumentationTok}[1]{\textcolor[rgb]{0.56,0.35,0.01}{\textbf{\textit{#1}}}}
\newcommand{\ErrorTok}[1]{\textcolor[rgb]{0.64,0.00,0.00}{\textbf{#1}}}
\newcommand{\ExtensionTok}[1]{#1}
\newcommand{\FloatTok}[1]{\textcolor[rgb]{0.00,0.00,0.81}{#1}}
\newcommand{\FunctionTok}[1]{\textcolor[rgb]{0.00,0.00,0.00}{#1}}
\newcommand{\ImportTok}[1]{#1}
\newcommand{\InformationTok}[1]{\textcolor[rgb]{0.56,0.35,0.01}{\textbf{\textit{#1}}}}
\newcommand{\KeywordTok}[1]{\textcolor[rgb]{0.13,0.29,0.53}{\textbf{#1}}}
\newcommand{\NormalTok}[1]{#1}
\newcommand{\OperatorTok}[1]{\textcolor[rgb]{0.81,0.36,0.00}{\textbf{#1}}}
\newcommand{\OtherTok}[1]{\textcolor[rgb]{0.56,0.35,0.01}{#1}}
\newcommand{\PreprocessorTok}[1]{\textcolor[rgb]{0.56,0.35,0.01}{\textit{#1}}}
\newcommand{\RegionMarkerTok}[1]{#1}
\newcommand{\SpecialCharTok}[1]{\textcolor[rgb]{0.00,0.00,0.00}{#1}}
\newcommand{\SpecialStringTok}[1]{\textcolor[rgb]{0.31,0.60,0.02}{#1}}
\newcommand{\StringTok}[1]{\textcolor[rgb]{0.31,0.60,0.02}{#1}}
\newcommand{\VariableTok}[1]{\textcolor[rgb]{0.00,0.00,0.00}{#1}}
\newcommand{\VerbatimStringTok}[1]{\textcolor[rgb]{0.31,0.60,0.02}{#1}}
\newcommand{\WarningTok}[1]{\textcolor[rgb]{0.56,0.35,0.01}{\textbf{\textit{#1}}}}
\usepackage{longtable,booktabs,array}
\usepackage{calc} % for calculating minipage widths
% Correct order of tables after \paragraph or \subparagraph
\usepackage{etoolbox}
\makeatletter
\patchcmd\longtable{\par}{\if@noskipsec\mbox{}\fi\par}{}{}
\makeatother
% Allow footnotes in longtable head/foot
\IfFileExists{footnotehyper.sty}{\usepackage{footnotehyper}}{\usepackage{footnote}}
\makesavenoteenv{longtable}
\usepackage{graphicx}
\makeatletter
\def\maxwidth{\ifdim\Gin@nat@width>\linewidth\linewidth\else\Gin@nat@width\fi}
\def\maxheight{\ifdim\Gin@nat@height>\textheight\textheight\else\Gin@nat@height\fi}
\makeatother
% Scale images if necessary, so that they will not overflow the page
% margins by default, and it is still possible to overwrite the defaults
% using explicit options in \includegraphics[width, height, ...]{}
\setkeys{Gin}{width=\maxwidth,height=\maxheight,keepaspectratio}
% Set default figure placement to htbp
\makeatletter
\def\fps@figure{htbp}
\makeatother
\setlength{\emergencystretch}{3em} % prevent overfull lines
\providecommand{\tightlist}{%
  \setlength{\itemsep}{0pt}\setlength{\parskip}{0pt}}
\setcounter{secnumdepth}{5}
\newlength{\cslhangindent}
\setlength{\cslhangindent}{1.5em}
\newlength{\csllabelwidth}
\setlength{\csllabelwidth}{3em}
\newlength{\cslentryspacingunit} % times entry-spacing
\setlength{\cslentryspacingunit}{\parskip}
\newenvironment{CSLReferences}[2] % #1 hanging-ident, #2 entry spacing
 {% don't indent paragraphs
  \setlength{\parindent}{0pt}
  % turn on hanging indent if param 1 is 1
  \ifodd #1
  \let\oldpar\par
  \def\par{\hangindent=\cslhangindent\oldpar}
  \fi
  % set entry spacing
  \setlength{\parskip}{#2\cslentryspacingunit}
 }%
 {}
\usepackage{calc}
\newcommand{\CSLBlock}[1]{#1\hfill\break}
\newcommand{\CSLLeftMargin}[1]{\parbox[t]{\csllabelwidth}{#1}}
\newcommand{\CSLRightInline}[1]{\parbox[t]{\linewidth - \csllabelwidth}{#1}\break}
\newcommand{\CSLIndent}[1]{\hspace{\cslhangindent}#1}
\ifLuaTeX
  \usepackage{selnolig}  % disable illegal ligatures
\fi

\begin{document}
\maketitle
\begin{abstract}
First sentence. Second sentence. Third sentence. Fourth sentence.
\end{abstract}

\hypertarget{introduction-talk-about-dataset-findings-and-conclusion}{%
\section{Introduction (talk about dataset, findings, and conclusion)}\label{introduction-talk-about-dataset-findings-and-conclusion}}

\hypertarget{st-paragraph-motivation-and-broad}{%
\section{1st paragraph: motivation and broad}\label{st-paragraph-motivation-and-broad}}

\hypertarget{nd-paragraph-what-i-did-and-findings}{%
\section{2nd paragraph: what I did and findings}\label{nd-paragraph-what-i-did-and-findings}}

\hypertarget{paragraph-implications}{%
\section{3 paragraph: implications}\label{paragraph-implications}}

Fire incidents occur everyday around the world. In most societies, these incidents are a source of threat to safety, capital, buildings, and societal development. They occur in different locations, with different causes, and result in damages to different extends.

This paper provides an in-depth analysis on the fire incidents occurred in cooking areas or kitchens located in Toronto, ON. Specifically, the author is interested in exploring the differences in estimated money loss as a result of the fire incidents, categorized by whether the fire alarm system was operated. The author's hypothesis is that fire incidents where the fire alarm system did not operate or did not exist would result in higher estimated dollar loss compared to those where the fire alarm system operated. Such a hypothesis

The author found that\ldots{}

The remainder of the paper will follow the structure of: Section \ref{data} explains the data; Section \ref{Model} showcases the models to support the analysis; Section \ref{Results} covers the results of the models; Section\ldots{}

\hypertarget{data-generate-a-table-and-talk-about-the-variables}{%
\section{Data (generate a table and talk about the variables)}\label{data-generate-a-table-and-talk-about-the-variables}}

The dataset is obtained from Open Data Toronto (Gelfand 2020) using R (R Core Team 2020). The dataset provides information on 17536 fire incidents to which Toronto Fire responds in more detail, as displayed by the 43 variables including Area of Origin, Extend of Fire, and Estimated Dollar Loss. The dataset includes only fire incidents defined by the Ontario Fire Marshal, which means it does not capture all fire incidents in Toronto. Since the fire incidents recorded are actual events related to citizens of Toronto and the society as a whole, personal information is not provided and the exact address have been approximated to the nearest intersections for privacy purposes. Moreover, the dataset follows exemptions under Section 8 of Municipal Freedom of Information and Protection of Privacy Act (MFIPPA) and excludes certain incidents.

For the purpose of the paper, the analysis will focus on the fire incidents occurred in cooking areas or kitchens. Moreover, the author is interested in exploring the relationship between the estimated dollar loss and the fire alarm system operation among the selected kitchen fire incidents. To do so, the author uses Tidyverse (Wickham et al. 2019) to select the variables of ``Area of Origin,'' ``Fire Alarm System Operation,'' ``Latitude,'' ``Longitude,'' and ``Estimated Dollar Loss.'' The author also removes all entrants with missing values for any of the variables. The following shows an extract of the cleaned dataset (Table \ref{tab:dataextract}).

The first variable, Area of Origin, shows the initial cause of the fire incidents. In the context of this paper, all observations should contain the Area of Origin of ``24 - Cooking Area or Kitchen.'' The second variable of Fire Alarm System Operation shows the status of the fire alarm system during the fire incident. The four statuses include: Fire alarm system operated; Fire alarm system did not operate; not applicable (no system); and Fire alarm system operation undetermined. The next two variables, longitude and latitude, are used in mapping the locations of the fire incidents. Lastly, the Estimated Dollar Loss showcases the financial consequences of each fire incident.

\begin{Shaded}
\begin{Highlighting}[]
\NormalTok{fire\_incidents }\OtherTok{\textless{}{-}}
  \FunctionTok{read\_csv}\NormalTok{(}\StringTok{"\textasciitilde{}/Session1/STA304\_Paper\_1/inputs/data/fire\_incidents.csv"}\NormalTok{)}
\end{Highlighting}
\end{Shaded}

\begin{verbatim}
## Rows: 17536 Columns: 43
\end{verbatim}

\begin{verbatim}
## -- Column specification --------------------------------------------------------
## Delimiter: ","
## chr  (25): Area_of_Origin, Building_Status, Business_Impact, Extent_Of_Fire,...
## dbl  (13): _id, Civilian_Casualties, Count_of_Persons_Rescued, Estimated_Dol...
## dttm  (5): Ext_agent_app_or_defer_time, Fire_Under_Control_Time, Last_TFS_Un...
\end{verbatim}

\begin{verbatim}
## 
## i Use `spec()` to retrieve the full column specification for this data.
## i Specify the column types or set `show_col_types = FALSE` to quiet this message.
\end{verbatim}

\begin{Shaded}
\begin{Highlighting}[]
\NormalTok{cleaned\_data }\OtherTok{\textless{}{-}}
\NormalTok{  fire\_incidents }\SpecialCharTok{\%\textgreater{}\%}
  \FunctionTok{select}\NormalTok{(Area\_of\_Origin, Fire\_Alarm\_System\_Operation, Latitude, Longitude, Estimated\_Dollar\_Loss) }\SpecialCharTok{\%\textgreater{}\%}
  \FunctionTok{filter}\NormalTok{(Area\_of\_Origin }\SpecialCharTok{==} \StringTok{"24 {-} Cooking Area or Kitchen"}\NormalTok{) }\SpecialCharTok{\%\textgreater{}\%}
  \FunctionTok{filter}\NormalTok{(}\SpecialCharTok{!}\FunctionTok{is.na}\NormalTok{(Fire\_Alarm\_System\_Operation)) }\SpecialCharTok{\%\textgreater{}\%}
  \FunctionTok{filter}\NormalTok{(}\SpecialCharTok{!}\FunctionTok{is.na}\NormalTok{(Latitude)) }\SpecialCharTok{\%\textgreater{}\%}
  \FunctionTok{filter}\NormalTok{(}\SpecialCharTok{!}\FunctionTok{is.na}\NormalTok{(Longitude)) }\SpecialCharTok{\%\textgreater{}\%}
  \FunctionTok{filter}\NormalTok{(}\SpecialCharTok{!}\FunctionTok{is.na}\NormalTok{(Estimated\_Dollar\_Loss))}

\NormalTok{data\_extract }\OtherTok{\textless{}{-}}
\NormalTok{  cleaned\_data }\SpecialCharTok{\%\textgreater{}\%} 
    \FunctionTok{slice}\NormalTok{(}\DecValTok{1}\SpecialCharTok{:}\DecValTok{10}\NormalTok{) }\SpecialCharTok{\%\textgreater{}\%}
      \FunctionTok{kable}\NormalTok{(}
      \AttributeTok{caption =} \StringTok{"First ten rows of a dataset of fire incidents in Toronto"}\NormalTok{,}
      \AttributeTok{col.names =} \FunctionTok{c}\NormalTok{(}\StringTok{"Area of Origin"}\NormalTok{, }\StringTok{"Fire Alarm System Operation"}\NormalTok{, }\StringTok{"Latitude"}\NormalTok{, }\StringTok{"Longitude"}\NormalTok{, }\StringTok{"Estimated Dollar Loss"}\NormalTok{),}
      \AttributeTok{digits =} \DecValTok{1}\NormalTok{,}
      \AttributeTok{booktabs =} \ConstantTok{TRUE}\NormalTok{, }
      \AttributeTok{linesep =} \StringTok{""}
\NormalTok{    )}
\end{Highlighting}
\end{Shaded}

\hypertarget{model}{%
\section{Model}\label{model}}

\hypertarget{results}{%
\section{Results}\label{results}}

\hypertarget{discussion}{%
\section{Discussion}\label{discussion}}

\hypertarget{first-discussion-point}{%
\subsection{First discussion point}\label{first-discussion-point}}

If my paper were 10 pages, then should be be at least 2.5 pages. The discussion is a chance to show off what you know and what you learnt from all this.

\hypertarget{second-discussion-point}{%
\subsection{Second discussion point}\label{second-discussion-point}}

\hypertarget{third-discussion-point}{%
\subsection{Third discussion point}\label{third-discussion-point}}

\hypertarget{weaknesses-and-next-steps}{%
\subsection{Weaknesses and next steps}\label{weaknesses-and-next-steps}}

Weaknesses and next steps should also be included.

\newpage

\appendix

\hypertarget{appendix}{%
\section*{Appendix}\label{appendix}}
\addcontentsline{toc}{section}{Appendix}

\hypertarget{additional-details}{%
\section{Additional details}\label{additional-details}}

\newpage

\hypertarget{references}{%
\section*{References}\label{references}}
\addcontentsline{toc}{section}{References}

\hypertarget{refs}{}
\begin{CSLReferences}{1}{0}
\leavevmode\vadjust pre{\hypertarget{ref-citeopendatatoronto}{}}%
Gelfand, Sharla. 2020. \emph{Opendatatoronto: Access the City of Toronto Open Data Portal}. \url{https://CRAN.R-project.org/package=opendatatoronto}.

\leavevmode\vadjust pre{\hypertarget{ref-citeR}{}}%
R Core Team. 2020. \emph{R: A Language and Environment for Statistical Computing}. Vienna, Austria: R Foundation for Statistical Computing. \url{https://www.R-project.org/}.

\leavevmode\vadjust pre{\hypertarget{ref-citetidyverse}{}}%
Wickham, Hadley, Mara Averick, Jennifer Bryan, Winston Chang, Lucy D'Agostino McGowan, Romain François, Garrett Grolemund, et al. 2019. {``Welcome to the {tidyverse}.''} \emph{Journal of Open Source Software} 4 (43): 1686. \url{https://doi.org/10.21105/joss.01686}.

\end{CSLReferences}

\end{document}
